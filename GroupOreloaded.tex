\documentclass[11pt]{article}
\usepackage{graphicx}
\begin{document}


\begin{figure}        
\paragraph{} 
 \begin{center}
 \huge \textbf{MAKERERE} \includegraphics[width=0.1\textwidth]{Muk_logo}  \huge \textbf{UNIVERSITY}
 \end{center}
\begin{center}
\huge  College Of Computing And Info. Sciences.
\end{center}
\begin{center}
\huge School Of Comp. And Informatics Tech.
\end{center}
\begin{center}
\huge Department Of Computer Science
\end{center}
\begin{center}
\huge BIT 2207 Research Methodology
\end{center}
\begin{center}
\huge GROUP ASSIGNMENT I\\
\end{center} 
\huge Lecturer: Ernest Mwebaze 
\end{figure}
\begin{tabular}{|c|l|c|l|c|}
\hline
& \textbf{Name} & \textbf{Std Number} & \textbf{Reg Number} & \textbf{Signature} \\ \hline
$1.$& \textbf{KONGORO} Dickens & 216020616 & {16/U/18805} &\ \\ \hline
$2.$& \textbf{AGABA} Davis & 216009915 & {16/U/2812/PS} &\\ \hline
$3.$& \textbf{ANKUNDA} Dorothy & 216014145 & {16/U/3483/PS} &\\ \hline
$4.$& \textbf{NALUTAAYA} Viola & 216014557 & {16/U/9075/PS} &\\ \hline

\end{tabular} 


\begin{titlepage}
\centerline{THE STRUCTURE OF YOUTH UNEMPLOYMENT IN UGANDA\\}
\paragraph*{•}
\centerline{  Prepared by:  GroupOreloaded.\\}
\paragraph*{•}
\paragraph*{•}
  \begin{flushright}
  The Report,\\
  DATE: $February,8^{th},2018$.
  \end{flushright}
\date{\today}
\end{titlepage}
\tableofcontents
\newpage
\section{Problem Statement}
\paragraph{}Youth in Uganda are the youngest population in the world, with 77$\%$ of its population being under 30 years of age. In Uganda, the male to female ratio is 100.2 males per 100 females. Life expectancy at birth for males is 42.59 years and 44.49 years for females. Ugandan youth experience different lifestyles depending on if they live in a rural area or urban area.
\paragraph{•}The unemployment rate for young people ages 15–24 is 83$\%$
Informal work accounts for the majority of young workers in Uganda. 3.2$\%$ of youth work for waged employment, 90.9$\%$ work for informal employment, and 5.8$\%$ of the Ugandan youth are self-employed.. This rate is even higher for those who have formal degrees and live in the urban area. This is due to the disconnect between the degree achieved and the vocational skills needed for the jobs that are in demand for workers.


\section{Major Objectives}
\begin{itemize}
\item To gain familiarity about youth unemployment in the country.
\item To understand the state and the structure of the youth unemployment in Uganda.
\item To study and get to know the causes of unemployment amongst the youths in Uganda.
\item To understand the major jobs that the youths are engaged in.
\item To establish an environment for youth to be exposed to the society full of unemployment and advise them accordingly. 
\item To make comparisons on the ratio of male to female unemployment.
\end{itemize}
\section{Minor Objectives}
\begin{itemize}
\item Desire to understand the employment status so that after campus/university you get to know the next course of action.
\end{itemize}
\newpage
\section{Significance of the study}
\begin{itemize}
\item To narrow the gap of unemployment amongst the youth.
\item To sensitize the youth on the causes of unemployment and the possible ways of cabbing it.
\item Widen employment opportunities.

\end{itemize}
\section{Methodology Used}
It is an Applied Research as our main aim is to solve a problem within the society(And that is Unemployment amongst the youth).
\end{document}